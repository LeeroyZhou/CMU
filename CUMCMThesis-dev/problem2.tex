\subsection{双目标优化模型的建立}
\subsubsection{目标函数确定}
通过对于问题的分析,新的任务定价方案与旧的定价方案相比,应该满足两个优化条件:成本更少、任务完成率更高。所以,本文将新的定价方案设计问题转化为双目标优化问题,目标函数如下:$$\left\{\begin{array}{l}
        \min \sum_{i=1}^{835} p_{i} \\
        \max \sum_{i=1}^{1835} C_{i}
    \end{array}\right.$$
其中,$p_i$为第$i$个任务的定价;$C_i$为1时,表示任务$i$被完成,$C_i$为0时,表示任务$i$未被完成。
\subsubsection{吸引度矩阵的建立}
Step1:吸引度的定义公式

在问题一未完成原因的分析求解中,本文得到了距离这个影响因子的影响作用最为显著。这表明会员在选择是否预定并完成某一单时,会重点考虑这一任务与自己的距离。同时,任务的定价直接影响到会员的收益,也会作为重要的考虑因素。

为了综合考虑这两个因素对于任务完成情况的影响,我们引入吸引度矩阵$W_{ij}$,其中矩阵中每个元素$w_{ij}$表示第$i$个任务对第$i$个任务对第$j$个会员的吸引度$(i=1,2\ldots 875,j=1,2\ldots 1875)$当任务距离会员越远,任务定价越低时,任务对会员的吸引度越低。具体定义如下:
$$w_{i j}=\sqrt{\frac{a}{l_{i j}^{2}}+b p_{i}^{2}}, \quad(\mathrm{i}=1,2 \cdots 875, \quad j=1,2 \cdots 1875)$$
其中,$l_{ij}$表示任务$i$与会员$j$之间的距离;$p_i$表示任务$i$的定价

我们假设吸引度$w_{ij}$的取值范围[0,1].。当吸引度的值接近0时,说明这个任
务对用户毫无吸引力;当吸引度的值接近于1时,说明这个任务非常具有吸引力。

Step2:参数a,b的求解

在问题一中我们曾对任务的位置分布做聚类分析,依据任务的位置信息将其
分为四类,每一类都有一个中心点。中心点的任务占据着绝对的地理优势,它对
与之距离最近的会员应当具有绝对大的吸引度,我们假设这个吸引度值为0.99。
在四个位于中心点的任务里选取两个任务,并找到与之最近的一个会员,计算两
者之间的距离,得到如下两组数据:
% Please add the following required packages to your document preamble:
% \usepackage{longtable}
% Note: It may be necessary to compile the document several times to get a multi-page table to line up properly
\begin{longtable}[c]{c|cc}
    \caption{选取的任务}
    \label{tab:my-table}       \\
    \hline
    任务号码 & A0716  & A0396  \\
    \endfirsthead
    %
    \endhead
    %
    \hline
    \endfoot
    %
    \endlastfoot
    %
    会员编号 & B0509  & B0972  \\
    $I_{ij}$ & 0.4514 & 0.2061 \\
    $P_i$    & 75     & 65.5   \\ \hline
\end{longtable}

将这两组数据代入表达式中,可以求得$a=0.01175,b=0.000164$

\subsubsection{任务吸引度的阈值确定}
在得到任务吸引度矩阵后,我们需要将每个任务对每个会员的吸引度$w_{ij}$ 与这
个任务的吸引度阈值$w_i$作比较,用以判断这个任务是否具备足够的吸引度被完
成。当吸引度大于等于阈值时,该任务具备足够的吸引度被会员领取并完成,即:
$$C_{i}=\left\{\begin{array}{ll}
        1, & w_{i j}>w_{i}      \\
        0, & w_{i j} \leq w_{i}
    \end{array}\right.$$

利用吸引度$w_{ij}$的计算公式与附件一中的任务数据计算出吸引度矩阵,当第$i$个任务被完成时,其阈值至少低于其对一个会员的吸引度;当第$i$个任务未被完成时,其阈值不低于任何其对一个会员的吸引度。阈值是作为一个判别指标与吸引度的值作比较,它的绝对值的大小并无实际意义,重要的是其相对大小,因此,阈值的确定具有一定的主观性。在这里,我们依据附件一中的任务数据,做出如下假设:
$$w_{i}=\left\{\begin{array}{l}
        \min \left\{w_{i j}\right\}, C_{i}=1 \\
        \max \left\{w_{i j}\right\}, C_{i}=0
    \end{array}\right.$$

\subsubsection{约束条件的确定}
在确定约束条件之前,我们结合题目中的条件给出如下准则:
\begin{enumerate}
    \item 最大吸引力准则:$\max \{w_{ij}\}$
    \item 竞争准则:$\max\{w_{ij}\}$
    \item 分配准则:$\text {belong}(i)=j \mid \begin{array}{l}
                  \max \left\{w_{i j}\right\} \\
                  \max \left\{G\left(n_{m}\right)\right\}
              \end{array}$,表示任务$i$分配给会员$j$
    \item 时间列准则:以第一个时间点6:30为例
          $$\begin{array}{l}
                  \qquad \operatorname{choice1}(j)=i \mid \begin{array}{l}
                      \max \left\{w_{i j}\right\}      \\
                      i=1,2 \cdots 835                 \\
                      j=1,2 \cdots, 7,14, \cdots, 1253 \\
                      \text { belong }(i)
                  \end{array} \\
              \end{array}$$
          约束条件为:
          $$\begin{array}{c}
                  50 \leq p_{i} \leq 100                                        \\
                  C_{i}=\left\{\begin{array}{l}
                      1, \quad \mathrm{W}_{i j}>w_{i} \\
                      0, \quad \mathrm{w}_{i j} \leq w_{i}
                  \end{array}\right.                 \\
                  w_{i j}=\sqrt{\frac{0.01175}{l_{i j}^{2}}+0.000164 p_{i}^{2}} \\
                  w_{i}=\left\{\begin{array}{c}
                      \min \left\{w_{i j}\right\}, C_{i}=1 \\
                      \max \left\{w_{i j}\right\}, C_{i}=0
                  \end{array}\right.                 \\
                  \operatorname{choice}(j)=k \mid \begin{array}{c}
                      w_{kj}=\min \left\{w_{i j}\right\} \\
                      i=1,2 \cdots 835, \quad j=1,2 \cdots 1875
                  \end{array}     \\
                  \begin{array}{c}
                      \qquad \operatorname{belong}(k)=n_{j} \mid \begin{array}{l}
                          w_{k n_{m}}=\max \left\{w_{i n_{m}}\right\},(m=1,2 \cdots n) \\
                          G\left(n_{j}\right)=\max \left\{G\left(n_{m}\right)\right\}(1 \leq j \leq n)
                      \end{array}
                  \end{array}
              \end{array}
          $$
\end{enumerate}

关于约束条件的说明如下:

1是关于任务定价的限定范围,这个价格范围是参照附件一中的任务定价区间大致给出的。

2表明当任务$i $对会员$j $的吸引度大于任务$i $的阈值时,任务$i$ 会被完成;当
任务$i$ 对会员j 的吸引度不大于任务$i$  的阈值时,任务$i$不会被完成。

3是上文计算出的吸引度矩阵中元素的计算公式。

4是每个任务的吸引度阈值的确定。

5是为了说明当会员$j $选择预定任务$k $时, $k $对该会员的吸引度在所有的待
选任务中是最大的,即会员会选择预约对自己吸引度最大的任务。

6是为了表明不同的会员在选择同一个任务时,信誉值最大的会员具有最大
优先度。如果有$n $个会员同时预定任务$k$ ,任务$k $最后被这$n $个会员中信誉值最
大的人成功预约。$G( j)$表示会员 $j$的信誉值。

所以,总的优化模型为:


$$\begin{aligned}
         & \begin{array}{l}
            \min \sum_{i=1}^{835} p_{i} \\
            \max \sum_{i=1}^{1835} C_{i}
        \end{array} \\
         & s.t.\quad
        \begin{cases}
             & 50 \leq p_{i} \leq 100                                        \\
             & C_{i}=\left\{\begin{array}{ll}
                1, & w_{i j}>w_{i}      \\
                0, & w_{i j} \leq w_{i}
            \end{array}\right.                \\
             & w_{i j}=\sqrt{\frac{0.01175}{l_{i j}^{2}}+0.000164 p_{i}^{2}} \\
             & w_{i}=\left\{\begin{array}{l}
                \min \left\{w_{i j}\right\}, C_{i}=1 \\
                \max \left\{w_{i j}\right\}, C_{i}=0
            \end{array}\right.                \\
             & \text {belong}(\mathrm{i})=j \mid \begin{array}{l}
                \max \left\{w_{i j}\right\} \\
                \max \left\{G\left(n_{m}\right)\right\}
            \end{array}  \\
             & \begin{array}{ll}
                \text {choice1}(j)=i \mid & \begin{array}{l}
                    \max \left\{w_{i j}\right\}      \\
                    \mathrm{i}=1,2 \cdots 835        \\
                    j=1,2 \cdots, 7,14, \cdots, 1253 \\
                    \text { belong }(\mathrm{i})
                \end{array}
            \end{array}                                    \\
            .                                                                \\
            .                                                                \\
            .                                                                \\
             & \begin{array}{l}
                \operatorname{choice} 31(j)=k\mid \begin{array}{l}
                    \max \left\{w_{i j}\right\} \\
                    j=153,165,169, \cdots, 1875 \\
                    \text { belong }(\mathrm{i})
                \end{array}
            \end{array}.
        \end{cases}
    \end{aligned}
$$
\subsection{双目标优化模型的求解}
\subsubsection{模型求解分析}
新的任务定价方案模型是在全面考虑约束条件下,建立的大型双目标优化模
型。模型的的解即长度为835 的一维定价矩阵,是通过竞争、分配、时间列准则
等个各个约束的限定,得到的满足目标函数的全局最优解。

但是模型的最优解的长度较长,约束条件复杂,如吸引度阈值、吸引度等各
个约束变量都是维度较大的矩阵,一方面即使在约束条件下对两个目标分别求解
是比较困难的,很大程度的原因在于模型中变量太多,尤其是模型的约束条件中
包含了时间列的动态变化准则,这种限制使得模型的变成非线性而且不易求解的
复杂数学模型;另一方面,从算法角度考虑, 为了得到最优的任务定价方案,
需要对任务的定价在一定范围内进行遍历并作为最外层循环,同时在内部也有吸
引度、阈值等大型数据矩阵以及内层循环遍历,进而使得算法的复杂度程指数上
升,这对算法运行时所需要的时间资源和内存资源存在很大要求。

因此综合考虑算法复杂度以及程序的运行实现,采用分布逐级优化的策略,
在算法中对模型中的约束进行一定简化,并做出适当假设,在存在一定误差之下,
得到全局最优解的近似解作为模型最优解,即任务定价方案。
\subsubsection{模型求解步骤}
Step1:预先设置任务定价

将任务的定价直接进行设定,并进行分组对比,通过得到的任务成本和任务
完成率两项指标,从而确定一种局部最优解,作为最终近似解。

Step2:吸引度矩阵确定

通过计算会员与每个任务之间的距离,根据模型二和问题一,由吸引度计算
公式得到每个任务对于每个会员的吸引度矩阵。

Step3:时间刻准则

满足不同时刻不同情况的约束条件,通过建立for 循环,将6:30 时刻转换
为编号1,依次递推,至8:00 编号31,通过循环遍历,满足不同时间可选择预
定人数不同的约束。

Step4:目标任务确定

从时刻编号1 开始,通过对释放能够预定任务的会员,相应的在该时刻能够
预定的任务中,通过遍历吸引度矩阵,找到最大吸引的任务,并记录位置。

Step5:冲突判断

判断位置记录矩阵中,是否存在数值相等的情况,即判断是否存在冲突。若
发生冲突,则对会员的信誉值进行比较,从而确定一个优先选择,即信誉高的会
员得到此次任务预定权。

Step5:方案及完成度结果

通过对时间和任务预定数的遍历,得到最终任务完成度矩阵,并输出任务完
成度数值,同时输出预先设定的任务定价矩阵。考虑到对复杂度和运行时间降低,
不采用对定价进行遍历的方式,而通过对任务定价矩阵的不同设置得到几组不同
结果,从而对比分析得到近似最优解。

\subsubsection{模型求解结果}

